\documentclass[a4paper,10pt]{article}
\usepackage{mathtext}
\usepackage[T2A]{fontenc}
\usepackage[utf8x]{inputenc}
\usepackage[english,ukrainian]{babel}
\usepackage{amsfonts,amsmath,stmaryrd}
\usepackage{indentfirst}
\usepackage[pdftex]{color,graphicx}

\DeclareGraphicsRule{.jpeg}{.bmp}{}{}

%opening
\title{Індуктивні методи моделювання навігації агента}
\author{Накрийко Андрій}

\begin{document}
\selectlanguage{ukrainian}
\maketitle

\begin{abstract}

\end{abstract}

\section{Вступ}
\section{Проблема керування}
\section{Експертні правила та штучна нейронна мережа}
Першим підходом до розв'язання задачі керування агентом було використання нейромережі з певним наперед заданим набором експертних правил. Ідея такого підходу полягає в тому, щоб, задавши набір ключових правил, використати здатність нейромережі до генералізації даних і поширити експертні знання на увесь простір можливих ситуацій. Таким чином, можна отримати контролер, який здатний приймати ``осмислені'' рішення в ситуаціях, для яких не було надано експертних правил, на основі правил, заданих для ``близьких'' ситуацій. 
\subsection{Результати підходу на основі експертних знань}

Результати не можна охарактеризувати однозначно. З однiєї сторони, система контролю на базi нейромережi досить адекватно керувала автомобiлем з доволі складною фізикою, успiшно оминаючи перешкоди та забезпечуючи безперервний рух автомобiля в середовищi, про яке вона не мала жодної попередньої iнформацiї. Бiльше того, для пiдвищення рiвня “природностi” керування, системi контролю надавалася лише iнформацiя про об’єкти, якi попадали в поле зору автомобiля, що являло собою сектор з визначеним радiусом (дальнiстю бачення) та центральним кутом (кутом огляду). Базуючись лише на даних про вiддаль до найближчих об’єктiв в полi зору, система контролю забезпечувала завчасне оминання перешкоди з плавною змiною швидкостi та напрямку руху автомобiля. Очевидно, що велику роль в досягненнi вказаної природностi та адекватностi керування вiдiграла вдало пiдiбрана навчальна множина.
     
З iншого боку, найбiльшою проблемою, з якою довелося зiткнутися, була наявнiсть ситуацiй, в яких автомобiль зупинявся i система контролю не могла вивести його з нерухомого стану. Це, зокрема, ситуацiя, подана на рис.~\ref{car_stuck}, в якiй автомобiль заїхав у глухий кут i застряг, не зумiвши здiйснити розворот назад. Цю проблему, теоретично, можна було б розв'язати, ввівши у визначення стану автомобіля значення поточної швидкості та задавши додаткові набори ключових експертних правил, які б дозволили автомобілю здійснити розвертання у разі застрягання в глухому куті. Проте на практиці це достатньо проблематично, оскільки введення додаткової змінної стану призводить до ускладення поведінки всієї системи, підвищення рівня вимог до точності та репрезентативності експертних правил. Більше того, для того, щоб зробити розвертання автомобіля безпечнішим та природнішим, довелося б вводити також змінні, які б відповідали за задній огляд автомобіля, що призводить до ще більш строгих вимог до експертних правил.

Слiд зазначити, що ще однiєю причиною (окрiм вказаної обмеженостi визначеного стану динамiчної системи) такої поведiнки є особливiсть фiзичної моделi автомобiля. Мається на увазi природа автомобiля "--- для того щоб здiйснити поворот, чи, тим бiльше, розворот, автомобiль повинен здiйснювати поступальний рух, що в умовах, коли вже вiдбулося ``застрягання'', дуже проблематично. Можливим розв’язанням даної проблеми є змiна типу транспортного засобу. Якщо взяти транспортний засiб, здатний здiйснювати поворот без поступального руху, то можна значно зменшити ймовiрнiсть його застрягання. Таким транспортним засобом, для прикладу, може бути танк, в якому шляхом незалежного обертання гусениць
в рiзнi сторони можна домогтися розвороту на будь-який кут, стоячи при цьому на мiсцi.

Ще однією важливою проблемою є вибір внутрішньої структури нейромережі. Якщо використати недостатню кількість нейронів, то нейромережа не зможе в достатній мірі вивчити набір правил і, таким чином, не зможе в повністю використати експертні знання. З іншого боку, використовуючи надто велику кількість нейронів, існує загроза надто точного запам'ятовування (overfitting) правил без належної генералізації їх на схожі ситуації. В такому випадку нейромережа буде точно виконувати задані правила у відповідних ситуаціях, але навіть незначна зміна ситуації призведе до різкої зміни значень керованих змінних, порівняно з близькою еталонною ситуацією. 

Таким чином, при використанні наперед заданих експертних правил, з'являється велика кількість практичних питань, на які немає чітких теоретичних відповідей, а все доводиться вирішувати в результаті численних експериментів. Саме тому було вирішено відійти від підходу, який базується на заздалегідь відібраних експертних знаннях, а піти шляхом самоорганізації "--- дати можливість агенту розробити власну систему правил на основі отриманого внаслідок взаємодії з середовищем досвіду.

\section{Самоорганізаційний підхід "--- навчання з підсиленням}
\subsection{Загальні поняття навчання з підсиленням}
\subsection{Класифікація методів розв'язку}
\subsection{Ускладнення задачі}
\section{Висновки}
\section{Література}
\begin{thebibliography}{99}
\bibitem{Kun03}{Philippe Kunzle. Vehicle Control with Neural Networks "--- http://www.gamedev.net/reference/articles/article1988.asp}
\bibitem{Wik08}{Wikipedia. Control Theory "--- http://en.wikipedia.org/wiki/Control\_theory}
\end{thebibliography}

\end{document}
